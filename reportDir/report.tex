%% bare_jrnl.tex
%% V1.4a
%% 2014/09/17
%% by Michael Shell
%% see http://www.michaelshell.org/
%% for current contact information.
%%
%% This is a skeleton file demonstrating the use of IEEEtran.cls
%% (requires IEEEtran.cls version 1.8a or later) with an IEEE
%% journal paper.
%%
%% Support sites:
%% http://www.michaelshell.org/tex/ieeetran/
%% http://www.ctan.org/tex-archive/macros/latex/contrib/IEEEtran/
%% and
%% http://www.ieee.org/

%%*************************************************************************
%% Legal Notice:
%% This code is offered as-is without any warranty either expressed or
%% implied; without even the implied warranty of MERCHANTABILITY or
%% FITNESS FOR A PARTICULAR PURPOSE! 
%% User assumes all risk.
%% In no event shall IEEE or any contributor to this code be liable for
%% any damages or losses, including, but not limited to, incidental,
%% consequential, or any other damages, resulting from the use or misuse
%% of any information contained here.
%%
%% All comments are the opinions of their respective authors and are not
%% necessarily endorsed by the IEEE.
%%
%% This work is distributed under the LaTeX Project Public License (LPPL)
%% ( http://www.latex-project.org/ ) version 1.3, and may be freely used,
%% distributed and modified. A copy of the LPPL, version 1.3, is included
%% in the base LaTeX documentation of all distributions of LaTeX released
%% 2003/12/01 or later.
%% Retain all contribution notices and credits.
%% ** Modified files should be clearly indicated as such, including  **
%% ** renaming them and changing author support contact information. **
%%
%% File list of work: IEEEtran.cls, IEEEtran_HOWTO.pdf, bare_adv.tex,
%%                    bare_conf.tex, bare_jrnl.tex, bare_conf_compsoc.tex,
%%                    bare_jrnl_compsoc.tex, bare_jrnl_transmag.tex
%%*************************************************************************


% *** Authors should verify (and, if needed, correct) their LaTeX system  ***
% *** with the testflow diagnostic prior to trusting their LaTeX platform ***
% *** with production work. IEEE's font choices and paper sizes can       ***
% *** trigger bugs that do not appear when using other class files.       ***                          ***
% The testflow support page is at:
% http://www.michaelshell.org/tex/testflow/



\documentclass[journal]{IEEEtran}
%
% If IEEEtran.cls has not been installed into the LaTeX system files,
% manually specify the path to it like:
% \documentclass[journal]{../sty/IEEEtran}





% Some very useful LaTeX packages include:
% (uncomment the ones you want to load)


% *** MISC UTILITY PACKAGES ***
%
%\usepackage{ifpdf}
% Heiko Oberdiek's ifpdf.sty is very useful if you need conditional
% compilation based on whether the output is pdf or dvi.
% usage:
% \ifpdf
%   % pdf code
% \else
%   % dvi code
% \fi
% The latest version of ifpdf.sty can be obtained from:
% http://www.ctan.org/tex-archive/macros/latex/contrib/oberdiek/
% Also, note that IEEEtran.cls V1.7 and later provides a builtin
% \ifCLASSINFOpdf conditional that works the same way.
% When switching from latex to pdflatex and vice-versa, the compiler may
% have to be run twice to clear warning/error messages.






% *** CITATION PACKAGES ***
%
%\usepackage{cite}
% cite.sty was written by Donald Arseneau
% V1.6 and later of IEEEtran pre-defines the format of the cite.sty package
% \cite{} output to follow that of IEEE. Loading the cite package will
% result in citation numbers being automatically sorted and properly
% "compressed/ranged". e.g., [1], [9], [2], [7], [5], [6] without using
% cite.sty will become [1], [2], [5]--[7], [9] using cite.sty. cite.sty's
% \cite will automatically add leading space, if needed. Use cite.sty's
% noadjust option (cite.sty V3.8 and later) if you want to turn this off
% such as if a citation ever needs to be enclosed in parenthesis.
% cite.sty is already installed on most LaTeX systems. Be sure and use
% version 5.0 (2009-03-20) and later if using hyperref.sty.
% The latest version can be obtained at:
% http://www.ctan.org/tex-archive/macros/latex/contrib/cite/
% The documentation is contained in the cite.sty file itself.






% *** GRAPHICS RELATED PACKAGES ***
%
\ifCLASSINFOpdf
  % \usepackage[pdftex]{graphicx}
  % declare the path(s) where your graphic files are
  % \graphicspath{{../pdf/}{../jpeg/}}
  % and their extensions so you won't have to specify these with
  % every instance of \includegraphics
  % \DeclareGraphicsExtensions{.pdf,.jpeg,.png}
\else
  % or other class option (dvipsone, dvipdf, if not using dvips). graphicx
  % will default to the driver specified in the system graphics.cfg if no
  % driver is specified.
  % \usepackage[dvips]{graphicx}
  % declare the path(s) where your graphic files are
  % \graphicspath{{../eps/}}
  % and their extensions so you won't have to specify these with
  % every instance of \includegraphics
  % \DeclareGraphicsExtensions{.eps}
\fi
% graphicx was written by David Carlisle and Sebastian Rahtz. It is
% required if you want graphics, photos, etc. graphicx.sty is already
% installed on most LaTeX systems. The latest version and documentation
% can be obtained at: 
% http://www.ctan.org/tex-archive/macros/latex/required/graphics/
% Another good source of documentation is "Using Imported Graphics in
% LaTeX2e" by Keith Reckdahl which can be found at:
% http://www.ctan.org/tex-archive/info/epslatex/
%
% latex, and pdflatex in dvi mode, support graphics in encapsulated
% postscript (.eps) format. pdflatex in pdf mode supports graphics
% in .pdf, .jpeg, .png and .mps (metapost) formats. Users should ensure
% that all non-photo figures use a vector format (.eps, .pdf, .mps) and
% not a bitmapped formats (.jpeg, .png). IEEE frowns on bitmapped formats
% which can result in "jaggedy"/blurry rendering of lines and letters as
% well as large increases in file sizes.
%
% You can find documentation about the pdfTeX application at:
% http://www.tug.org/applications/pdftex





% *** MATH PACKAGES ***
%
%\usepackage[cmex10]{amsmath}
% A popular package from the American Mathematical Society that provides
% many useful and powerful commands for dealing with mathematics. If using
% it, be sure to load this package with the cmex10 option to ensure that
% only type 1 fonts will utilized at all point sizes. Without this option,
% it is possible that some math symbols, particularly those within
% footnotes, will be rendered in bitmap form which will result in a
% document that can not be IEEE Xplore compliant!
%
% Also, note that the amsmath package sets \interdisplaylinepenalty to 10000
% thus preventing page breaks from occurring within multiline equations. Use:
%\interdisplaylinepenalty=2500
% after loading amsmath to restore such page breaks as IEEEtran.cls normally
% does. amsmath.sty is already installed on most LaTeX systems. The latest
% version and documentation can be obtained at:
% http://www.ctan.org/tex-archive/macros/latex/required/amslatex/math/





% *** SPECIALIZED LIST PACKAGES ***
%
%\usepackage{algorithmic}
% algorithmic.sty was written by Peter Williams and Rogerio Brito.
% This package provides an algorithmic environment fo describing algorithms.
% You can use the algorithmic environment in-text or within a figure
% environment to provide for a floating algorithm. Do NOT use the algorithm
% floating environment provided by algorithm.sty (by the same authors) or
% algorithm2e.sty (by Christophe Fiorio) as IEEE does not use dedicated
% algorithm float types and packages that provide these will not provide
% correct IEEE style captions. The latest version and documentation of
% algorithmic.sty can be obtained at:
% http://www.ctan.org/tex-archive/macros/latex/contrib/algorithms/
% There is also a support site at:
% http://algorithms.berlios.de/index.html
% Also of interest may be the (relatively newer and more customizable)
% algorithmicx.sty package by Szasz Janos:
% http://www.ctan.org/tex-archive/macros/latex/contrib/algorithmicx/




% *** ALIGNMENT PACKAGES ***
%
%\usepackage{array}
% Frank Mittelbach's and David Carlisle's array.sty patches and improves
% the standard LaTeX2e array and tabular environments to provide better
% appearance and additional user controls. As the default LaTeX2e table
% generation code is lacking to the point of almost being broken with
% respect to the quality of the end results, all users are strongly
% advised to use an enhanced (at the very least that provided by array.sty)
% set of table tools. array.sty is already installed on most systems. The
% latest version and documentation can be obtained at:
% http://www.ctan.org/tex-archive/macros/latex/required/tools/


% IEEEtran contains the IEEEeqnarray family of commands that can be used to
% generate multiline equations as well as matrices, tables, etc., of high
% quality.




% *** SUBFIGURE PACKAGES ***
%\ifCLASSOPTIONcompsoc
%  \usepackage[caption=false,font=normalsize,labelfont=sf,textfont=sf]{subfig}
%\else
%  \usepackage[caption=false,font=footnotesize]{subfig}
%\fi
% subfig.sty, written by Steven Douglas Cochran, is the modern replacement
% for subfigure.sty, the latter of which is no longer maintained and is
% incompatible with some LaTeX packages including fixltx2e. However,
% subfig.sty requires and automatically loads Axel Sommerfeldt's caption.sty
% which will override IEEEtran.cls' handling of captions and this will result
% in non-IEEE style figure/table captions. To prevent this problem, be sure
% and invoke subfig.sty's "caption=false" package option (available since
% subfig.sty version 1.3, 2005/06/28) as this is will preserve IEEEtran.cls
% handling of captions.
% Note that the Computer Society format requires a larger sans serif font
% than the serif footnote size font used in traditional IEEE formatting
% and thus the need to invoke different subfig.sty package options depending
% on whether compsoc mode has been enabled.
%
% The latest version and documentation of subfig.sty can be obtained at:
% http://www.ctan.org/tex-archive/macros/latex/contrib/subfig/




% *** FLOAT PACKAGES ***
%
%\usepackage{fixltx2e}
% fixltx2e, the successor to the earlier fix2col.sty, was written by
% Frank Mittelbach and David Carlisle. This package corrects a few problems
% in the LaTeX2e kernel, the most notable of which is that in current
% LaTeX2e releases, the ordering of single and double column floats is not
% guaranteed to be preserved. Thus, an unpatched LaTeX2e can allow a
% single column figure to be placed prior to an earlier double column
% figure. The latest version and documentation can be found at:
% http://www.ctan.org/tex-archive/macros/latex/base/


%\usepackage{stfloats}
% stfloats.sty was written by Sigitas Tolusis. This package gives LaTeX2e
% the ability to do double column floats at the bottom of the page as well
% as the top. (e.g., "\begin{figure*}[!b]" is not normally possible in
% LaTeX2e). It also provides a command:
%\fnbelowfloat
% to enable the placement of footnotes below bottom floats (the standard
% LaTeX2e kernel puts them above bottom floats). This is an invasive package
% which rewrites many portions of the LaTeX2e float routines. It may not work
% with other packages that modify the LaTeX2e float routines. The latest
% version and documentation can be obtained at:
% http://www.ctan.org/tex-archive/macros/latex/contrib/sttools/
% Do not use the stfloats baselinefloat ability as IEEE does not allow
% \baselineskip to stretch. Authors submitting work to the IEEE should note
% that IEEE rarely uses double column equations and that authors should try
% to avoid such use. Do not be tempted to use the cuted.sty or midfloat.sty
% packages (also by Sigitas Tolusis) as IEEE does not format its papers in
% such ways.
% Do not attempt to use stfloats with fixltx2e as they are incompatible.
% Instead, use Morten Hogholm'a dblfloatfix which combines the features
% of both fixltx2e and stfloats:
%
% \usepackage{dblfloatfix}
% The latest version can be found at:
% http://www.ctan.org/tex-archive/macros/latex/contrib/dblfloatfix/




%\ifCLASSOPTIONcaptionsoff
%  \usepackage[nomarkers]{endfloat}
% \let\MYoriglatexcaption\caption
% \renewcommand{\caption}[2][\relax]{\MYoriglatexcaption[#2]{#2}}
%\fi
% endfloat.sty was written by James Darrell McCauley, Jeff Goldberg and 
% Axel Sommerfeldt. This package may be useful when used in conjunction with 
% IEEEtran.cls'  captionsoff option. Some IEEE journals/societies require that
% submissions have lists of figures/tables at the end of the paper and that
% figures/tables without any captions are placed on a page by themselves at
% the end of the document. If needed, the draftcls IEEEtran class option or
% \CLASSINPUTbaselinestretch interface can be used to increase the line
% spacing as well. Be sure and use the nomarkers option of endfloat to
% prevent endfloat from "marking" where the figures would have been placed
% in the text. The two hack lines of code above are a slight modification of
% that suggested by in the endfloat docs (section 8.4.1) to ensure that
% the full captions always appear in the list of figures/tables - even if
% the user used the short optional argument of \caption[]{}.
% IEEE papers do not typically make use of \caption[]'s optional argument,
% so this should not be an issue. A similar trick can be used to disable
% captions of packages such as subfig.sty that lack options to turn off
% the subcaptions:
% For subfig.sty:
% \let\MYorigsubfloat\subfloat
% \renewcommand{\subfloat}[2][\relax]{\MYorigsubfloat[]{#2}}
% However, the above trick will not work if both optional arguments of
% the \subfloat command are used. Furthermore, there needs to be a
% description of each subfigure *somewhere* and endfloat does not add
% subfigure captions to its list of figures. Thus, the best approach is to
% avoid the use of subfigure captions (many IEEE journals avoid them anyway)
% and instead reference/explain all the subfigures within the main caption.
% The latest version of endfloat.sty and its documentation can obtained at:
% http://www.ctan.org/tex-archive/macros/latex/contrib/endfloat/
%
% The IEEEtran \ifCLASSOPTIONcaptionsoff conditional can also be used
% later in the document, say, to conditionally put the References on a 
% page by themselves.




% *** PDF, URL AND HYPERLINK PACKAGES ***
%
%\usepackage{url}
% url.sty was written by Donald Arseneau. It provides better support for
% handling and breaking URLs. url.sty is already installed on most LaTeX
% systems. The latest version and documentation can be obtained at:
% http://www.ctan.org/tex-archive/macros/latex/contrib/url/
% Basically, \url{my_url_here}.




% *** Do not adjust lengths that control margins, column widths, etc. ***
% *** Do not use packages that alter fonts (such as pslatex).         ***
% There should be no need to do such things with IEEEtran.cls V1.6 and later.
% (Unless specifically asked to do so by the journal or conference you plan
% to submit to, of course. )


% correct bad hyphenation here
\hyphenation{op-tical net-works semi-conduc-tor}


\begin{document}
%
% paper title
% Titles are generally capitalized except for words such as a, an, and, as,
% at, but, by, for, in, nor, of, on, or, the, to and up, which are usually
% not capitalized unless they are the first or last word of the title.
% Linebreaks \\ can be used within to get better formatting as desired.
% Do not put math or special symbols in the title.
\title{Comparison of Clustering Algorithms for Searching in High-Dimensional Data}
%
%
% author names and IEEE memberships
% note positions of commas and nonbreaking spaces ( ~ ) LaTeX will not break
% a structure at a ~ so this keeps an author's name from being broken across
% two lines.
% use \thanks{} to gain access to the first footnote area
% a separate \thanks must be used for each paragraph as LaTeX2e's \thanks
% was not built to handle multiple paragraphs
%

\author{Daniel Dunning \rule{1in}{0pt} \and Yu Zhuang}
    

% note the % following the last \IEEEmembership and also \thanks - 
% these prevent an unwanted space from occurring between the last author name
% and the end of the author line. i.e., if you had this:
% 
% \author{....lastname \thanks{...} \thanks{...} }
%                     ^------------^------------^----Do not want these spaces!
%
% a space would be appended to the last name and could cause every name on that
% line to be shifted left slightly. This is one of those "LaTeX things". For
% instance, "\textbf{A} \textbf{B}" will typeset as "A B" not "AB". To get
% "AB" then you have to do: "\textbf{A}\textbf{B}"
% \thanks is no different in this regard, so shield the last } of each \thanks
% that ends a line with a % and do not let a space in before the next \thanks.
% Spaces after \IEEEmembership other than the last one are OK (and needed) as
% you are supposed to have spaces between the names. For what it is worth,
% this is a minor point as most people would not even notice if the said evil
% space somehow managed to creep in.



% The paper headers
% The only time the second header will appear is for the odd numbered pages
% after the title page when using the twoside option.
% 
% *** Note that you probably will NOT want to include the author's ***
% *** name in the headers of peer review papers.                   ***
% You can use \ifCLASSOPTIONpeerreview for conditional compilation here if
% you desire.




% If you want to put a publisher's ID mark on the page you can do it like
% this:
%\IEEEpubid{0000--0000/00\$00.00~\copyright~2014 IEEE}
% Remember, if you use this you must call \IEEEpubidadjcol in the second
% column for its text to clear the IEEEpubid mark.



% use for special paper notices
%\IEEEspecialpapernotice{(Invited Paper)}




% make the title area
\maketitle

% As a general rule, do not put math, special symbols or citations
% in the abstract or keywords.
\begin{abstract}
Searching data is an area of research that is always open to improvement. The problem of searching one-dimensional data has already been solved, but with the increase of Big Data, as well as an increase in the need for computational sciences, higher-dimensional data has been on the rise. It is known that searching sorted data is exceptionally faster than unsorted, but the problem has become finding efficient methods of sorting data which has more than one dimension. The purpose of this paper is not to create a new method for sorting high-dimensional data, rather, comparing previous methods and comparing their efficiency. In addition, we will investigate the searching methods performed on each the clustered data, and study whether this should have any impact on the use of the clustering algorithms.
\end{abstract}

% Note that keywords are not normally used for peerreview papers.
\begin{IEEEkeywords}
Clustering, High-dimension data, Searching, Efficiency
\end{IEEEkeywords}






% For peer review papers, you can put extra information on the cover
% page as needed:
% \ifCLASSOPTIONpeerreview
% \begin{center} \bfseries EDICS Category: 3-BBND \end{center}
% \fi
%
% For peerreview papers, this IEEEtran command inserts a page break and
% creates the second title. It will be ignored for other modes.
\IEEEpeerreviewmaketitle



\section{Introduction}
% The very first letter is a 2 line initial drop letter followed
% by the rest of the first word in caps.
% 
% form to use if the first word consists of a single letter:
% \IEEEPARstart{A}{demo} file is ....
% 
% form to use if you need the single drop letter followed by
% normal text (unknown if ever used by IEEE):
% \IEEEPARstart{A}{}demo file is ....
% 
% Some journals put the first two words in caps:
% \IEEEPARstart{T}{his demo} file is ....
% 
% Here we have the typical use of a "T" for an initial drop letter
% and "HIS" in caps to complete the first word.
\IEEEPARstart{T}{here} has constantly been a need for searching data in all fields of research. This task can be either searching for a specific point in the data set, or searching for a point which is closest to a given query point. No scientific or technological research area can avoid the need to compare new data with already gathered data, so the desire to do this efficiently is high. The techniques for actually searching data that is sorted are mostly known; the most famous example being a binary search of a list. Initially, the question must be asked, whether it is computationally worth it to sort data before it is searched. The search time of a brute force search, searching through all n elements of a data set, is \textit{O(n)}. Going back to the example, the best case for search time for a binary search is: the time taken to sort the data (at best \textit{O(nlog(n))}) plus the binary search time (\textit{log(n)}). This total time is more than a simple brute force search, so why bother with sorting the data? The reason lies in the fact that sorting allows us to search the same data set efficiently for multiple queries. If a particular data is only ever search once and then discarded, it would be a waste to perform a sorting technique to it before search, because it would take more time. However, this is rarely the case, and in most cases data sets need to be searched for thousands of queries, making sorting the efficient solution.\par

Sorting techniques have been researched extensively for one-dimensional data, and the concept of sorting, for example, a list of numbers is simple: the numbers are arranged in ascending (or descending) order. However, as we increase dimensions, the task of grouping the data in terms of \textit{least} and \textit{most} becomes harder. Even using a two-dimensional example, it is easy to say that on a coordinate plane a point with small x and small y values is "smaller" than a number with large x and large y values. However, how do we categorize when x is small and y is large, or x is large and y is small? These distinctions are harder to make, so instead we \textit{cluster} the data by how similar it is to those points around it. We can then perform groupings within the data, and can use these groups or clusters to search data points that are similar. \par

The remainder the the paper will follow as such. Section 2 will discuss the problem in more detail and give an explanation for each of the data sets used. Section 3 will give more detail to explain the clustering algorithms and the search procedure used within them. Section 4 will discuss the procedure of the research and the experiments that followed. Finally, Section 5 will discuss the results of the tests and Section 6 will discuss the contributions of this research and future research.

\section{Problem and Data Sets}
The main goal of this research was to compare various methods of clustering high-dimensional data, and study the efficiency of them, both for their speed in clustering, but also in the speed of the searching for the closest data point in each cluster using the resulting clusters. In addition to having a dimensionality of at least two, the data sets were also expected to have a sufficiently large amount data points attributed to them. Four data sets were used for the tests. In the tests performed, up to 10,000 query points were randomly generated (with the corresponding dimensions based on their data set counterpart). The seed for these random query points was kept the same through all algorithms and their subsequent tests, so that the results were comparable. The first data set was randomly generated and not used for testing the efficiency of the algorithms, as it was relatively small, with only 10,000 data points of two dimensions, so there was little time distinction between the algorithms using this set. This data set was generated by having mostly random points, but with the addition of several points per query that were close in distance to each query point. This is done as mentioned in REFERENCE HERE because in most real-world scenarios there are a couple of points which lie very close the the query, and this data set was created to mimic those scenarios. In addition, the data set was used to check for algorithm accuracy compared to a brute force search, and also to show a two-dimensional representation of how the clustering algorithms worked. The second data set is a list a the longitude and latitude of all major cities in the world. The third set consists of the results from an experiment studying the chemical reactions between ethylene gas and carbon-dioxide gas in a chamber, with the results coming from several sensors over the course of many time steps. The last data set comes from several studies of cancerous cells, which contains the gene expression of all possible genes taken from cancer cells. These data sets were chosen because of the unique characteristics each one brings; The world coordinates contain millions of data points with low dimensionality, the chemical reaction set has millions of points while also have fairly high dimensionality, and the cancer set has a small number of points with extremely high dimensionality. The data sets are listed in Table 1 with their respective attributes.
INSERT TABLE HERE WITH THE 4 DATA SETS

\section{Algorithms}
The clustering algorithms that were used for the research are KD-Tree, KMeans, and Locality Sensitive Hashing (LSH). These are three of the primary clustering algorithms used for data mining, and they all cluster the data using different techniques.

\subsection{KD-Tree}
KD-Tree is a recursive algorithm used to cluster data in rectangle-shaped groups. This is done by splitting the data set in half along the dimension with the highest variance. GIVE EQUATION FOR VARIANCE. This splitting is then performed on each subsequent group until the desired number of clusters is reached. As a matter of principle, it is efficient to use a number of clusters equal to a power of two. Over the course of the clustering, it is important to keep track of each cluster's minimum and maximum value for each dimension, as this will be store at the end as the boundaries of the cluster. This boundary is used for searching the clusters. For each query, a euclidian distance will be calculated between the closest point of the cluster and the query point, which will be 0 if the point lies within the boundaries of the cluster. At this point, all the data points in the closest cluster are exhaustively searched to find the minimum distance to the query point. At this point, it is important to check whether there are any clusters which are closer to the query point than the minimum distance found. This is because it is possible for a point to have its nearest neighbor lie in a cluster different than the one the query point falls into. For every cluster that is closer than the initially-found distance, we exhaustively check every point in those clusters, comparing to the minimum distance. If the particular application in fact only needed an approximate nearest neighbor, and not an exact nearest neighbor, the last step in the search can be excluded. One would only need to search for the closest data point in the closest cluster, and that would be sufficient. An example of the clustering done using KD-Tree is shown in Figure 1. INSERT FIGURE

\subsection{KMeans}
KMeans was one of the first algorithms used for clustering multi-dimensional data. The process involves initializing centroids equal to the number of clusters needed. Each data point is then assigned to the centroid it is closest with, and a new centroid is calculated using the average value in each dimension of the points assigned to it. This process continues, assigning points to a centroid and calculating new centroids, until there is no change in centroid from one iteration to the next, or the assignment of data points does not change. The clusters that are formed take on a circular shape, and for this reason the cluster boundary uses the radius from each cluster centroid, which is simply the distance to the farthest point in each cluster. Searching can then be done by finding which cluster is the closest, and searching points in it. The closest distance will be the distance to the centroid, minus the cluster radius. If this value is negative, it means the query point falls inside of that cluster. As with KD-Tree, once the initial minimum distance to a point within a cluster is found, it is important to check points in other clusters where the distance to that cluster is less than the minimum distance to a point already found. \par
KMeans is a very powerful algorithm, and will of course always yield the correct results, however it is extremely computation-heavy. Each iteration requires each point finding the different lengths to each centroid, which is very time-consuming when there are a large number of centroids and the data has many dimensions. In addition, the number of iterations can be high, which leads to more computations. The amount of time that the algorithm takes is very dependent on the initial centroids. The simple way to generate them, apart from random points, is to choose a data point as the starting centroid. Next, choose the data point which is farthest away from that point as the next centroid. For the future centroids, obtain maximum distances to each centroid for each data point, and choose the minimum of those maximums. This technique, however, tends to not yield much better efficiency results than simply choosing random centroids. The technique used to calculate the initial centroids for this research is a variation of KMeans called Bisecting-KMeans. The initial centroid is chosen the same way, choosing a random data point. After the second point is chosen (the original way), however, the KMeans algorithm is then performed on those two points, assigning data to them, and calculating new centroids, with the same stopping conditions. Once the two centroids are "stable" i.e. unchanging, Then the next centroid is added, using the minimum maximum point method, and the three centroids and then put through the algorithm. It is obvious that the initialization of the centroids takes longer than the simple ways, because it must go through the KMeans algorithm at each step. However, hope of doing this is that each additional centroid will not take long to compute if the other ones are already "good" centroids, and the goal of this technique is that the actual KMeans algorithm will run more efficiently when the initial centroids start off close to their final spot. Figure 2 shows an example of clustering done by KMeans. INSERT FIGURE

% needed in second column of first page if using \IEEEpubid
%\IEEEpubidadjcol

\subsection{LSH}
The LSH algorithm REFERENCE HERE BOTH PAPERS is much more recent than either KD-Tree or KMeans, and it clusters data in a different way than its those algorithms. The simple goal of LSH is to obtain several hash values for each data point, which together make up its overall hash value. The idea is that data points which are near each other will share the same or similar (differing by one number) hash values. Getting the hash values involves using this formula INSERT FORMULA where INSERT DESCRIPTIONS. Apart from calculating the dot product, LSH benefits from being unaffected by the dimensionality of the data, because it mainly works with hash values for each data point, rather than the point itself. As one might expect, data points that have the same hash value are grouped in the same cluster. Another difference between LSH and the former algorithms is that the number of clusters is undefined. The number of clusters is dependent on how many vectors are used, the values of the vectors themselves, and how much the clusters are "cut" with the \textit{w} value. Searching is done by following the same hashing steps on the query point, and searching data points in the cluster with the matching hash value. As described in the papers, the searching is only an approximate search, and no information is given for an exact search after LSH has been applied. This is one novel part of this research and our contribution, where we give a solution for an exact search. It is more complicated to find the appropriate additional clusters to search because it depends heavily on the vectors used for hashing. In two dimensions the clusters tend to look like parallelograms, and the closer the vectors are to being parallel, the thinner the clusters become, meaning clusters farther away of potential of containing the closest point. The technique simply follows a combination of the previous approaches for the other algorithms. Boundaries are found in each dimension, and a centroid and radius is found, for each cluster. For each query point, we find which of these boundaries is the furthest, and associate that value with each cluster. The larger of the two distances is used when trying to find the minimum distance because we cannot afford to exclude a cluster, since the boundaries offer different cluster shapes. This results in slightly more computation, but guaranteed correctness. Once the maximum boundary distance for each cluster is found, the searching becomes the with previous algorithms. If that distance is less than the distance to a data point found, that cluster's points must also be searched. It should be noted that this process of finding boundaries and radii, along with the exact search adds a great amount of time to the algorithm, though it is necessary for proper comparison. An example of clustering done with \textit{m} = 3 and \textit{w} = 500 is shown in Figure 3.

\section{Test Procedure}
Testing of the algorithms was straightforward. Each algorithm was tested on each set of data, with different numbers of clusters in each test, so that the most efficient combination could be achieved for each algorithm. The time taken to cluster the data was measured independent from the time taken to search the data. For each algorithm the parameters that minimized the total clustering plus search time was used as a final comparison between the algorithms for each data set. It is important to measure the time for clustering and searching independently because one can gather a greater breadth of knowledge about the algorithm by investigating the efficiency with which it performs each task. Each algorithm had a specific set of requirements imposed by the researchers for how the number of clusters should be chosen. The number of clusters (\textit{k}) for KD-Tree was set to a power of 2. This is for simplicity and efficiency, as the algorithm works by splitting each previous cluster in half, so very little time is saved from not doubling the number of clusters at each step. For instance, instead of choosing 512 clusters, if one chooses 500 clusters, the clustering time is relatively unchanged and, except for particular data sets, the search time would be increased. The value for \textit{k} when running KMeans was unable to be increased above 50 in most cases. The reasoning for this is explained further, however the simple explanation is even when increasing the efficiency by combining it with Bisecting-kmeans, the shear amount of computation time with either large data sets or extremely high-dimensionality causes the algorithm to run longer than a reasonable amount (10+ hours) just for clustering. LSH is a different type of algorithm than the other two, in that the number of clusters is not specified by the developer. Instead, it is a bi-product of the number of vectors used and the way each vector is partitioned. Instead of choosing a specific number of clusters when running LSH, the researchers had to choose values of \textit{m} and \textit{w} such that an informative and reasonable number of clusters resulted. This involved some initial guess and check, partnered with more educated guess based on early tests.

\section{Results}
The graphs from the respective tests are shown in Figures 4 through SOMETHING. A similar trend for all data sets appears from the graphs. KMeans was the slowest algorithm for both clustering and searching. The comparison of searching is rather unfair for this algorithm, because with large data sets, when the data is only clustered into at most 50 clusters, there is still a high number of points to exhaustively search within that cluster, so not as much time is saved in the clustering. In fact, the searching techniques for both KD-Tree and KMeans is quite similar, so the KMeans algorithm could, in theory, perform similarly in searching. However, it falls drastically behind in its clustering efficiency, so much so that the searching cannot even be compared on the same level. Both KD-Tree and LSH had efficient clustering of all data sets. This due to the fact that, unlike KMeans, neither algorithm requires performing distance calculations for all the points, and comparing these distances, until the searching phase, which is a commonality for all algorithms. Instead, each algorithm only performs a single (in reality it is one calculation for each dimension) calculation for each cluster, and compares across that calculation as oppose to the calculations needed for each point in KMeans. The main calculation needed in KD-Tree is to compute the variance for each dimension in each cluster so that  it can split the cluster along the correct dimension. For LSH, the computation comes from computing the dot product with each data point and the vectors used for the hashing. \par

In most cases, KD-Tree had faster search times, while LSH presented drastically better clustering times. This is to be expected, as the characteristics needed to do a search on the clusters for both of the algorithms involves the boundaries of the clusters. KD-Tree keeps track of these values in each phase of its clustering. However, LSH does not typically need that information unless it intends to perform an exact search, so these values must be calculated during the search time for LSH. There needed to be a distinct time for calculating these borders needed for an exact search in LSH. In the other algorithms this time is part of the clustering, however, in LSH it was added to the search time. The reasoning behind this choice is the fact that the other algorithms need to keep track of the boundary values in clustering, so adding them to their appropriate lists or arrays is not time consuming, in fact, it is the exact same time as if it was done after clustering. For LSH, on the other hand, the boundary values are not added until after the clustering is completed because it is not needed for clustering. In fact, unlike the other two algorithms, those cluster boundaries are not even needed for an approximate search of the single closest cluster. It is only needed for the exact search in which other clusters need to be checked. Figure SOMETHING+1 shows each algorithms best time for each data set.

\section{Discussion}
The following section will discuss some contributions that were made as a result of this research, as well as some considerations for the future.
 \subsection{Contributions}
 
 
 \subsection{Future Work}


% An example of a floating figure using the graphicx package.
% Note that \label must occur AFTER (or within) \caption.
% For figures, \caption should occur after the \includegraphics.
% Note that IEEEtran v1.7 and later has special internal code that
% is designed to preserve the operation of \label within \caption
% even when the captionsoff option is in effect. However, because
% of issues like this, it may be the safest practice to put all your
% \label just after \caption rather than within \caption{}.
%
% Reminder: the "draftcls" or "draftclsnofoot", not "draft", class
% option should be used if it is desired that the figures are to be
% displayed while in draft mode.
%
%\begin{figure}[!t]
%\centering
%\includegraphics[width=2.5in]{myfigure}
% where an .eps filename suffix will be assumed under latex, 
% and a .pdf suffix will be assumed for pdflatex; or what has been declared
% via \DeclareGraphicsExtensions.
%\caption{Simulation results for the network.}
%\label{fig_sim}
%\end{figure}

% Note that IEEE typically puts floats only at the top, even when this
% results in a large percentage of a column being occupied by floats.


% An example of a double column floating figure using two subfigures.
% (The subfig.sty package must be loaded for this to work.)
% The subfigure \label commands are set within each subfloat command,
% and the \label for the overall figure must come after \caption.
% \hfil is used as a separator to get equal spacing.
% Watch out that the combined width of all the subfigures on a 
% line do not exceed the text width or a line break will occur.
%
%\begin{figure*}[!t]
%\centering
%\subfloat[Case I]{\includegraphics[width=2.5in]{box}%
%\label{fig_first_case}}
%\hfil
%\subfloat[Case II]{\includegraphics[width=2.5in]{box}%
%\label{fig_second_case}}
%\caption{Simulation results for the network.}
%\label{fig_sim}
%\end{figure*}
%
% Note that often IEEE papers with subfigures do not employ subfigure
% captions (using the optional argument to \subfloat[]), but instead will
% reference/describe all of them (a), (b), etc., within the main caption.
% Be aware that for subfig.sty to generate the (a), (b), etc., subfigure
% labels, the optional argument to \subfloat must be present. If a
% subcaption is not desired, just leave its contents blank,
% e.g., \subfloat[].


% An example of a floating table. Note that, for IEEE style tables, the
% \caption command should come BEFORE the table and, given that table
% captions serve much like titles, are usually capitalized except for words
% such as a, an, and, as, at, but, by, for, in, nor, of, on, or, the, to
% and up, which are usually not capitalized unless they are the first or
% last word of the caption. Table text will default to \footnotesize as
% IEEE normally uses this smaller font for tables.
% The \label must come after \caption as always.
%
%\begin{table}[!t]
%% increase table row spacing, adjust to taste
%\renewcommand{\arraystretch}{1.3}
% if using array.sty, it might be a good idea to tweak the value of
% \extrarowheight as needed to properly center the text within the cells
%\caption{An Example of a Table}
%\label{table_example}
%\centering
%% Some packages, such as MDW tools, offer better commands for making tables
%% than the plain LaTeX2e tabular which is used here.
%\begin{tabular}{|c||c|}
%\hline
%One & Two\\
%\hline
%Three & Four\\
%\hline
%\end{tabular}
%\end{table}


% Note that the IEEE does not put floats in the very first column
% - or typically anywhere on the first page for that matter. Also,
% in-text middle ("here") positioning is typically not used, but it
% is allowed and encouraged for Computer Society conferences (but
% not Computer Society journals). Most IEEE journals/conferences use
% top floats exclusively. 
% Note that, LaTeX2e, unlike IEEE journals/conferences, places
% footnotes above bottom floats. This can be corrected via the
% \fnbelowfloat command of the stfloats package.




\section{Conclusion}
The conclusion goes here.





% if have a single appendix:
%\appendix[Proof of the Zonklar Equations]
% or
%\appendix  % for no appendix heading
% do not use \section anymore after \appendix, only \section*
% is possibly needed

% use appendices with more than one appendix
% then use \section to start each appendix
% you must declare a \section before using any
% \subsection or using \label (\appendices by itself
% starts a section numbered zero.)
%


\appendices
\section{Proof of the First Zonklar Equation}
Appendix one text goes here.

% you can choose not to have a title for an appendix
% if you want by leaving the argument blank
\section{}
Appendix two text goes here.


% use section* for acknowledgment
\section*{Acknowledgment}


The authors would like to thank...


% Can use something like this to put references on a page
% by themselves when using endfloat and the captionsoff option.
\ifCLASSOPTIONcaptionsoff
  \newpage
\fi



% trigger a \newpage just before the given reference
% number - used to balance the columns on the last page
% adjust value as needed - may need to be readjusted if
% the document is modified later
%\IEEEtriggeratref{8}
% The "triggered" command can be changed if desired:
%\IEEEtriggercmd{\enlargethispage{-5in}}

% references section

% can use a bibliography generated by BibTeX as a .bbl file
% BibTeX documentation can be easily obtained at:
% http://www.ctan.org/tex-archive/biblio/bibtex/contrib/doc/
% The IEEEtran BibTeX style support page is at:
% http://www.michaelshell.org/tex/ieeetran/bibtex/
%\bibliographystyle{IEEEtran}
% argument is your BibTeX string definitions and bibliography database(s)
%\bibliography{IEEEabrv,../bib/paper}
%
% <OR> manually copy in the resultant .bbl file
% set second argument of \begin to the number of references
% (used to reserve space for the reference number labels box)
\begin{thebibliography}{1}

\bibitem{IEEEhowto:kopka}
H.~Kopka and P.~W. Daly, \emph{A Guide to \LaTeX}, 3rd~ed.\hskip 1em plus
  0.5em minus 0.4em\relax Harlow, England: Addison-Wesley, 1999.

\end{thebibliography}

% biography section
% 
% If you have an EPS/PDF photo (graphicx package needed) extra braces are
% needed around the contents of the optional argument to biography to prevent
% the LaTeX parser from getting confused when it sees the complicated
% \includegraphics command within an optional argument. (You could create
% your own custom macro containing the \includegraphics command to make things
% simpler here.)
%\begin{IEEEbiography}[{\includegraphics[width=1in,height=1.25in,clip,keepaspectratio]{mshell}}]{Michael Shell}
% or if you just want to reserve a space for a photo:

\begin{IEEEbiography}{Michael Shell}
Biography text here.
\end{IEEEbiography}

% if you will not have a photo at all:
\begin{IEEEbiographynophoto}{John Doe}
Biography text here.
\end{IEEEbiographynophoto}

% insert where needed to balance the two columns on the last page with
% biographies
%\newpage

\begin{IEEEbiographynophoto}{Jane Doe}
Biography text here.
\end{IEEEbiographynophoto}

% You can push biographies down or up by placing
% a \vfill before or after them. The appropriate
% use of \vfill depends on what kind of text is
% on the last page and whether or not the columns
% are being equalized.

%\vfill

% Can be used to pull up biographies so that the bottom of the last one
% is flush with the other column.
%\enlargethispage{-5in}



% that's all folks
\end{document}
